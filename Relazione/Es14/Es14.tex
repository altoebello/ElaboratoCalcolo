\section{Esercizio 14}

Eseguendo lo scrpit:
\lstinputlisting[language=Matlab]{CodiceMatlab/Esercizio14/scriptEs14.m}

Le operazioni:\\
-  $A \backslash b$: da come risultato il vettore x, che rappresenta la soluzione del sistema lineare $Ax = b$; Questa operazione da lo stesso risultato di  inv(A)*b se la matrice A ha rango massimo ed è non singolare.
\[\begin{bmatrix}
1 \\
2 \\
3 \\
4 \\
5\\
6 \\
7 \\
8\\
\end{bmatrix}\]

- $(A'*A) \backslash(A'* b)$ : poichè stiamo lavorando con una matrice mal condizionata, il vettore x risultante presenta un errore di approssimazione;
\[\begin{bmatrix}
3.5759 \\
-3.4624 \\
9.5151 \\
-1.2974\\
7.9574\\
4.9125\\
7.2378 \\
7.9765\\
\end{bmatrix}\]
\\

Le due espressioni sopra elencate eseguono la stessa operazione : risoluzione del sistema lineare $Ax = b$; La differenza fra i risultati è data dal fatto che la matrice di Vandermonde è malcondizionata : quando si eseguono moltiplicazioni fra matrici mal condizionate, il coefficiente di condizionamento aumenta e di conseguenza influenza il risultato ottenuto.
In questo esempio abbiamo che il condizionamento di A = 1.542832727600791e+09. Nel primo caso, quando andiamo ad effettuare $A \backslash b$, il condizionamento non è rilevante fino al punto di influenzare il risultato ottenuto. Nel secondo caso invece, dopo la moltiplicazione della matrice A con la sua trasposta, il valore di condizionamento risulta = 4.489735065328789e+18.Questo valore di condizionamento introdurrà un errore quando viene eseguita l'operazione successiva.


