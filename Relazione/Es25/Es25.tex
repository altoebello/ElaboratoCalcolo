\section{Esercizio 25}
\begin{itemize}
\item Simpson Adattiva
\lstinputlisting[language=Matlab]{CodiceMatlab/Esercizio25/adapsim.m}
\item Trapezi Adattiva
\lstinputlisting[language=Matlab]{CodiceMatlab/Esercizio25/adaptrap.m}
\item script
\lstinputlisting[language=Matlab]{CodiceMatlab/Esercizio25/scriptEs25.m}
\end{itemize}
Di seguito sono riportati i dati ottenuti:
\begin{table}[ht]
	\centering
	\small
	\begin{tabular}{| c | c | c| c | c | }
	\hline
	Tolleranza & Trapezio& N.Punti & Simpson & N. punti\\
	\hline
	$10^{-2}$ & 2.955597117841284e-01& 21 &2.812976430626699e-01&17\\
	\hline
	$10^{-3}$ &2.945853681850339e-01& 93 &2.812976430626699e-01&17\\
	\hline
	$10^{-4}$  &2.942742008736351e-01& 277 &2.942593384196308e-01&41\\
	\hline
	$10^{-5}$ &2.942301421648779e-01& 793 & 2.942278097680047e-01&81\\
	\hline
	$10^{-6}$ &2.942260196031783e-01& 2693&2.942257646203842e-01&145\\
	\hline
	\end{tabular}
\end{table}
\\$\int_{-1}^{1} \frac{1}{1+10^2x^2}  dx =0.29423$ ( circa ).\\
\\
Dai dati riportati in tabella risulta che entrambe le funzioni calcolano un'approssimazione molto precisa dell'integrale.\\
Inoltre,si nota immediatamente che la formula adattiva di Simpson calcola l'approssimazione dell'integrale utilizzando un numero di punti  nettamente inferiore  rispetto a quella del Trapezio.Il costo computazionale in questo caso dipende dal numero di chiamate ricorsive effettuate ( cioè di punti  calcolati ) , per cui possiamo affermare che  la formula adattiva di Simpson è più efficiente rispetto a quella del Trapezio.