
\section{Esercizio 6}
Dai risultati ottenuti possiamo affermare che il metodo di Newton sia il più efficiente ( in termini di iterazioni ), seguito dal metodo delle secanti e della bisezione ( che ha una crescita lineare ) e infine, dal metodo delle corde. Per quanto riguarda il costo computazionale, l'operazione che è decisamente più costosa è la valutazione funzionale delle ascisse tramite feval.\
\begin{itemize}
\item Bisezione\\
Per il metodo della bisezione la funzione è eseguita : una volta all'interno del ciclo e due volte fuori. Avremo quindi: 2+ n.iterazioni.
\item Newton \\
Per il metodo di Newton la funzione è eseguita due volte nel ciclo : 2 * n.iterazioni.
\item corde \\
Per il metodo delle corde funzione è eseguita una volta dentro il ciclo e una fuori : 1 + n.iterazioni.
\item secanti\\
Per il metodo delle secanti la funzione è eseguita una volta dentro il ciclo e una fuori : 1 + n.iterazioni.
\end{itemize}
Per tutte e 4 le funzioni, si ha un costo lineare in n.

\begin{table}[ht!]
	\centering
	\small
	\begin{tabular}{| c | c | c | }
	\hline
	Tolleranza & Iterazioni & Newton\\
	\hline
	 $10^{-3}$ & 4 & 7.390851333852840e-01\\
	\hline
	$10^{-6}$  & 5 & 7.390851332151607e-01\\
	\hline
	$10^{-9}$  &  5& 7.390851332151607e-01\\
	\hline
	$10^{-12}$ & 6 & 7.390851332151607e-01\\
	\hline
	\end{tabular}
\end{table}

\begin{table}[ht!]
	\centering
	\small
	\begin{tabular}{| c | c | c | c |}
	\hline
	Tolleranza & Iterazioni & Secanti\\
	\hline
	 $10^{-3}$ & 4 & 7.390851121274639e-01\\
	\hline
	$10^{-6}$ & 5 & 7.390851332150012e-01\\
	\hline
	$10^{-9}$ & 6 & 7.390851332151607e-01\\
	\hline
	$10^{-12}$ & 6 & 7.390851332151607e-01\\
	\hline
	\end{tabular}
\end{table}

\begin{table}[ht!]
	\centering
	\small
	\begin{tabular}{| c | c | c | }
	\hline
	Tolleranza & Iterazioni & Bisezione\\
	\hline
	 $10^{-3}$ & 10 & 7.392578125000000e-01\\
	\hline
	$10^{-6}$  & 20 &  7.390851974487305e-01\\
	\hline
	$10^{-9}$  &  30 &7.390851331874728e-01\\
	\hline
	$10^{-12}$ & 40 & 7.390851332156672e-01\\
	\hline
	\end{tabular}
\end{table}

\begin{table}[ht!]
	\centering
	\small
	\begin{tabular}{| c | c | c | }
	\hline
	Tolleranza & Iterazioni & Corde\\
	\hline
	 $10^{-3}$ & 17 & 7.395672022122561e-01\\
	\hline
	$10^{-6}$  & 34 &7.390845495752126e-01\\
	\hline
	$10^{-9}$  & 52 & 7.390851327392538e-01\\
	\hline
	$10^{-12}$ & 69 &7.390851332157368e-01\\
	\hline
	\end{tabular}
\end{table}



