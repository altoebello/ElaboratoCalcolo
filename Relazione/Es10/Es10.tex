\section{Esercizio 10}
\begin{itemize}
\item linsis
\lstinputlisting[language=Matlab]{CodiceMatlab/Esercizio10/linsis.m}
\item Script
\lstinputlisting[language=Matlab]{CodiceMatlab/Esercizio10/scriptEs10.m}
\end{itemize}
\begin{table}[ht]
	\centering
	\small
	\begin{tabular}{| c | c | c |}
	\hline
	Iterazione & Sigma & Norma\\
	\hline
	 1 & 0.1000 = $10^{-1}$ & 8.9839e-15\\
	\hline
	2 & 10 & 1.4865e-14\\
	\hline
	3 & 1000 = $10^{3}$ & 1.3712e-12\\
	\hline
	4 & 100000 =$10^{5}$ & 1.2948e-10\\
	\hline
	 5 & 10000000 = $10^{7}$ & 5.3084e-09\\
	\hline
	6 & $10^{9}$ & 1.0058e-06\\
	\hline 
	7 &  $10^{11}$ & 8.5643e-05\\
	\hline
	8&  $10^{13}$ & 0.0107\\
	\hline
	9&  $10^{15}$ & 0.9814\\
	\hline
	10&  $10^{17}$ &4.1004e+03\\
	\hline
	\end{tabular}
\end{table}

Possiamo notare che all'aumentare delle iterazioni il valore di sigma aumenta incrementa con un fattore di $10^{2}$, mentre la norma cresce all'incirca di $10^{1}$. Questo aumento della norma è causato dall'incremento del valore di condizionamento della matrice.