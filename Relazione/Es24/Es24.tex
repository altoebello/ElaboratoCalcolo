\section{Esercizio 24}
\begin{itemize}
\item Simpson Composita
\lstinputlisting[language=Matlab]{CodiceMatlab/Esercizio24/simpcomp.m}
\item Trapezi Composita
\lstinputlisting[language=Matlab]{CodiceMatlab/Esercizio24/trapecomp.m}
\item script
\lstinputlisting[language=Matlab]{CodiceMatlab/Esercizio24/scriptEs24.m}
\end{itemize}

Sono state riportate solo le prime 10 iterazioni.
\begin{table}[ht!]
	\centering
	\small
	\begin{tabular}{| c | c | c|}
	\hline
	Iterazioni & Trapezio & Simpson\\
	\hline
	2 & 2.664035584060345e-01&2.126315681335669e-01\\
	\hline
	4&2.034328044500163e-01&1.824425531313435e-01\\
	\hline
	6 &1.884983466139722e-01&1.773334438860330e-01\\
	\hline
	8&1.827894088752250e-01&1.759082770169613e-01\\
	\hline
	10&1.800348035219603e-01&1.753928683822888e-01 \\
	\hline
	12&1.785040157074719e-01&1.751725720719718e-01\\
	\hline
	14&1.775682181954108e-01&1.750665465192467e-01\\
	\hline
	16&1.769554131112014e-01&1.750107478565269e-01\\
	\hline
	18&1.765327096164695e-01&1.749792544399417e-01\\
	\hline
	20&1.762290375520298e-01&1.749604488953863e-01\\
	\hline
	\end{tabular}
\end{table}

$\int_{-1}^{1.1} tan(x)  dx = log(\frac{cos(1)}{cos(1.1)}) = 0.174922$ ( circa ) , per cui risulta ovvio che l'approssimazione dell'integrale risulta più precisa utilizzando il metodo di Simpson.\\
 Dal punto di vista computazionale la formula dei trapezi composita risulta leggermente meno costosa ( tutti e due i metodi  hanno costo lineare in n ): entrambi  eseguono una volta la funzione feval, ma il metodo dei trapezi composita calcola l'approssimazione dell'integrale eseguendo meno operazioni rispetto al metodo di Simpson.


