\section{Esercizio1}
Verificare che, per h sufficientemente piccolo, si ha \\

\begin{equation*}
\frac{f(x-h)-2f(x)+f(x+h)}{h^2} = O(h^2)
\end{equation*}
Per la dimostrazione utilizziamo il polinomio di taylor di grado n centrato in $x_0:$

\begin{gather*}
P_n(x;x_0) = \sum_{k=0}^{n}\frac{(x-x_0)^k}{k!}f^{(k)}(x_0)\\
R_n(x;x_0) = O(x-x_0)^{(n+1)}\\
\end{gather*}
Per cui possiamo calcolare $f(x+h)$ e $f(x-h)$ sviluppando il polinomio di Taylor centrato in x: \\
\begin{gather*}
f(x+h) = f(x)+hf'(x)+\frac{h^2}{2}f''(x)+\frac{h^3}{6}f'''(x)+O(h^4)\\
f(x-h) = f(x)-hf'(x)+\frac{h^2}{2}f''(x)-\frac{h^3}{6}f'''(x)+O(h^4)
\end{gather*}
Sostituendo all'equazione iniziale otteniamo: \\
\begin{equation*}	
\frac{f(x-h)-2f(x)+f(x+h)}{h^2}  = \frac{(h^2)f''(x)+O(h^4)}{h^2}=  f''(x)+O(h^2)
\end{equation*}
