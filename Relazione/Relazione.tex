\documentclass[12pt]{article}

\usepackage{amsmath}


\begin{document}
\title{Elaborato Calcolo numerico 2020\\  Autori: Emanuele Brizzi, Massimo Hong}
\maketitle
\newpage
\tableofcontents
\cleardoublepage

\section{Esercizio1}

Verificare che, per h sufficientemente piccolo, si ha \\

\begin{equation*}
\frac{f(x-h)-2f(x)+f(x+h)}{h^2} = O(h^2)
\end{equation*}
Per la dimostrazione utilizziamo il polinomio di taylor di grado n centrato in $x_0:$

\begin{gather*}
P_n(x;x_0) = \sum_{k=0}^{n}\frac{(x-x_0)^k}{k!}f^{(k)}(x_0)\\
R_n(x;x_0) = O(x-x_0)^{(n+1)}\\
\end{gather*}
Per cui possiamo calcolare $f(x+h)$ e $f(x-h)$ sviluppando il polinomio di Taylor centrato in x: \\
\begin{gather*}
f(x+h) = f(x)+hf'(x)+\frac{h^2}{2}f''(x)+\frac{h^3}{6}f'''(x)+O(h^4)\\
f(x-h) = f(x)-hf'(x)+\frac{h^2}{2}f''(x)-\frac{h^3}{6}f'''(x)+O(h^4)
\end{gather*}
Sostituendo all'equazione iniziale otteniamo: \\
\begin{equation*}	
\frac{f(x-h)-2f(x)+f(x+h)}{h^2}  = \frac{(h^2)f''(x)+O(h^4)}{h^2}=  f''(x)+O(h^2)
\end{equation*}



\section{Esercizio2}
Spiegare cosa calcola il seguente script Matlab:
u = 1; while 1, if 1+u==1, break, end, u = u/2; end, u

Teoricamente questo script non dovrebbe mai terminare poichè dividendo per 2 il valore u, non si dovrebbe mai raggiungere lo 0, tuttavia il seguente script termina. Più precisamente, alla fine dell'esecuzione  risulta che $u = 1.1102\cdot10^-16$ , che è minore del valore $d = eps = 2.2204\cdot10^-16$(che rappresenta la distanza da 1.0 al valore in doppia precisione immediatamente successivo) e quando andiamo a sommare ad un numero n un valore u<eps, non verrà effettuata alcuna modifica sul valore.

\section{Esercizio3}
\begin{enumerate}
\item a = 1e20; b = 100; a-a+b
\item a = 1e20; b = 100; a+b-a
\end{enumerate}
Spiegare i risultati ottenuti.
\begin{enumerate}
\item a= 1e20; b = 100; a-a+b \\
Questo script restituisce il valore 100, in quanto $a-a = 0$ e $0+100 = 100$.
\item a= 1e20; b = 100; a+b-a
Matlab ha il valore $ eps = 2.2204\cdot10^-16$ , che corrisponde a circa 15-16 cifre di precisione ( che 1e20 supera facilmente), per cui quando andiamo ad affettuare $a+b = c$, otterremo un valore approssimato c con le prime 15-16 cifre equivalenti ad a, e di conseguenza $c-a = 0$.
\end{enumerate}

\section{Esercizio 4}
A quanto pare si deve usare la formula di newton (Modificandola un po').

\section{Esercizio 6}
\begin{table}[ht]
	\centering
	\small
	\begin{tabular}{| c | c | c | c |}
	\hline
	Tolleranza & Bisezione & Newton & Secanti\\
	\hline
	 $10^{-3}$ & 7.392578125000000e-01 & 7.390851333852840e-01 & 7.390851121274639e-01\\
	\hline
	$10^{-6}$ & 7.390851974487305e-01 & 7.390851332151607e-01 & 7.390851332150012e-01\\
	\hline
	$10^{-9}$ & 7.390851331874728e-01 & 7.390851332151607e-01 & 7.390851332151607e-01\\
	\hline
	$10^{-12}$ & 7.390851332156672e-01 & 7.390851332151607e-01 & 7.390851332151607e-01\\
	\hline
	\end{tabular}
\end{table}


\begin{table}[ht]
	 \renewcommand\arraystretch{2}
	\centering
	\small
	\begin{tabular}{| c | c | }
	\hline
	Tolleranza & Corde\\
	\hline
	 $10^{-3}$ & 7.395672022122561e-01\\
	\hline
	$10^{-6}$  & 7.390845495752126e-01\\
	\hline
	$10^{-9}$  & 7.390851327392538e-01\\
	\hline
	$10^{-12}$ & 7.390851332157368e-01\\
	\hline
	\end{tabular}
\end{table}

\section{Esercizio 7}
La molteplicità m di f(x) = $x^2tan(x)$ è m=3. Sostituendo a x il valore zero otteniamo: $(0)^2*tan(0)$; 0 annulla due volte il primo termine del prodotto mentre annulla una volta il secondo termine, in quanto tan(0)=0;
\begin{table}[ht]
	\centering
	\small
	\begin{tabular}{| c | c | c | c |}
	\hline
	Tolleranza & Newton & Newton Modificato & Aitken\\
	\hline
	 $10^{-3}$ & 1.994002961956096e-03 & 6.617444900424221e-24 &-1.570796335324655e+00\\
	\hline
	$10^{-6}$ & 1.349222209381150e-06 & 6.617444900424221e-24 & -1.570796356741072e+00\\
	\hline
	$10^{-9}$ & 1.369405530548002e-09 & 0 & -1.570796314458764e+00\\
	\hline
	$10^{-12}$ & 1.389890778595252e-12 & 0 & Il metodo non converge\\
	\hline
	\end{tabular}
\end{table}
\section{Esercizio 10}

\begin{table}[ht]
	\centering
	\small
	\begin{tabular}{| c | c | c |}
	\hline
	Iterazione & Sigma & Norma\\
	\hline
	 1 & 0.1000 = $10^{-1}$ & 8.9839e-15\\
	\hline
	2 & 10 & 1.4865e-14\\
	\hline
	3 & 1000 = $10^{3}$ & 1.3712e-12\\
	\hline
	4 & 100000 =$10^{5}$ & 1.2948e-10\\
	\hline
	 5 & 10000000 = $10^{7}$ & 5.3084e-09\\
	\hline
	6 & $10^{9}$ & 1.0058e-06\\
	\hline 
	7 &  $10^{11}$ & 8.5643e-05\\
	\hline
	8&  $10^{13}$ & 0.0107\\
	\hline
	9&  $10^{15}$ & 0.9814\\
	\hline
	10&  $10^{17}$ &4.1004e+03\\
	\hline
	\end{tabular}
\end{table}
\section{Esercizio 13}
Eseguendo lo script:


A = [1 2 3 ; 1 2 4 ; 3 4 5 ; 3 4 6 ; 5 6 7];
b=[14 17 26 29 38];

QR = myqr(A);
disp(QR);
sol = qrsolve(QR,b);
disp(sol);

Otteniamo: 

La Matrice QR:
\[\begin{bmatrix}
-6.7082 & -8.6461  & -11.1803\\
0.1297 & -1.1155 & -2.9881\\
0.3892 & -0.0827 & 1.0351\\
 0.3892 & -0.0827  & -0.8037\\
0.6487 & -0.5222  & -0.4346\\

\end{bmatrix}\]

Soluzione del sistema lineare: 

\[\begin{bmatrix}
1\\ 2\\3\\

\end{bmatrix}\]
\section{Esercizio 14}

Dati i seguenti comandi:

A = rot90(vander(1:10));
A = A(:,1:8);
x = (1:8)'; 
b = A*x;

Le operazioni:\\
-  $A \backslash b$: da come risultato il vettore x, che rappresenta la soluzione del sistema lineare $Ax = b$; Questa operazione da lo stesso risultato di  inv(A)*b se la matrice A ha rango massimo ed è non singolare.
\[\begin{bmatrix}
1 \\
2 \\
3 \\
4 \\
5\\
6 \\
7 \\
8\\
\end{bmatrix}\]

- $(A'*A) \backslash(A'* b)$ : poichè stiamo lavorando con una matrice mal condizionata, il vettore x risultante presenta un errore di approssimazione;
\[\begin{bmatrix}
3.5759 \\
-3.4624 \\
9.5151 \\
-1.2974\\
7.9574\\
4.9125\\
7.2378 \\
7.9765\\
\end{bmatrix}\]
\\
Questo è il warning che segnala Matlab :\\
Warning: Matrix is close to singular or badly scaled. Results
may be inaccurate. RCOND =  2.393980e-19. 


\section{Esercizio 24}
Sono state riportate solo le prime 10 iterazioni.

\begin{table}[ht]
	\centering
	\small
	\begin{tabular}{| c | c | c|}
	\hline
	Iterazioni & Trapezio & Simpson\\
	\hline
	2 & 2.664035584060345e-01&2.126315681335669e-01\\
	\hline
	4&2.034328044500163e-01&1.824425531313435e-01\\
	\hline
	6 &1.884983466139722e-01&1.773334438860330e-01\\
	\hline
	8&1.827894088752250e-01&1.759082770169613e-01\\
	\hline
	10&1.800348035219603e-01&1.753928683822888e-01 \\
	\hline
	12&1.785040157074719e-01&1.751725720719718e-01\\
	\hline
	14&1.775682181954108e-01&1.750665465192467e-01\\
	\hline
	16&1.769554131112014e-01&1.750107478565269e-01\\
	\hline
	18&1.765327096164695e-01&1.749792544399417e-01\\
	\hline
	20&1.762290375520298e-01&1.749604488953863e-01\\
	\hline
	\end{tabular}
\end{table}




\section{Esercizio 25}
\begin{itemize}
\item Simpson Adattiva
\lstinputlisting[language=Matlab]{CodiceMatlab/Esercizio25/adapsim.m}
\item Trapezi Adattiva
\lstinputlisting[language=Matlab]{CodiceMatlab/Esercizio25/adaptrap.m}
\item script
\lstinputlisting[language=Matlab]{CodiceMatlab/Esercizio25/scriptEs25.m}
\end{itemize}
Di seguito sono riportati i dati ottenuti:
\begin{table}[ht]
	\centering
	\small
	\begin{tabular}{| c | c | c| c | c | }
	\hline
	Tolleranza & Trapezio& N.Punti & Simpson & N. punti\\
	\hline
	$10^{-2}$ & 2.955597117841284e-01& 21 &2.812976430626699e-01&17\\
	\hline
	$10^{-3}$ &2.945853681850339e-01& 93 &2.812976430626699e-01&17\\
	\hline
	$10^{-4}$  &2.942742008736351e-01& 277 &2.942593384196308e-01&41\\
	\hline
	$10^{-5}$ &2.942301421648779e-01& 793 & 2.942278097680047e-01&81\\
	\hline
	$10^{-6}$ &2.942260196031783e-01& 2693&2.942257646203842e-01&145\\
	\hline
	\end{tabular}
\end{table}
\\$\int_{-1}^{1} \frac{1}{1+10^2x^2}  dx =0.29423$ ( circa ).\\
\\
Dai dati riportati in tabella risulta che entrambe le funzioni calcolano un'approssimazione molto precisa dell'integrale.\\
Inoltre,si nota immediatamente che la formula adattiva di Simpson calcola l'approssimazione dell'integrale utilizzando un numero di punti  nettamente inferiore  rispetto a quella del Trapezio.Il costo computazionale in questo caso dipende dal numero di chiamate ricorsive effettuate ( cioè di punti  calcolati ) , per cui possiamo affermare che  la formula adattiva di Simpson è più efficiente rispetto a quella del Trapezio.
\end{document}