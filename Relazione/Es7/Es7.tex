\section{Esercizio 7}
La molteplicità m di f(x) = $x^2tan(x)$ è m=3. Sostituendo a x il valore zero otteniamo: $(0)^2*tan(0)$; 0 annulla due volte il primo termine del prodotto mentre annulla una volta il secondo termine, in quanto tan(0)=0;

\begin{itemize}
    \item Metodo di newtonModificato
    \lstinputlisting[language=Matlab]{CodiceMatlab/Esercizio7/newtonModificato.m}
    \item Metodo di Aitken
    \lstinputlisting[language=Matlab]{CodiceMatlab/Esercizio7/aitken.m}
  
\end{itemize}


\begin{table}[ht]
	\centering
	\small
	\begin{tabular}{| c | c | c |}
	\hline
	Tolleranza & Iterazioni & Newton\\
	\hline
	 $10^{-3}$  &16 & 1.994002961956096e-03 \\
	\hline
	$10^{-6}$ & 34&1.349222209381150e-06 \\
	\hline
	$10^{-9}$ &51& 1.369405530548002e-09 \\
	\hline
	$10^{-12}$ & 68 & 1.389890778595252e-12 \\
	\hline
	\end{tabular}
\end{table}


\begin{table}[ht]
	\centering
	\small
	\begin{tabular}{| c | c | c | c | c|}
	\hline
	Tolleranza & Iterazioni & Newton Modificato \\
	\hline
	 $10^{-3}$  & 4 & 6.617444900424221e-24 \\
	\hline
	$10^{-6}$ & 4 & 6.617444900424221e-24 \\
	\hline
	$10^{-9}$ &  5 & 0 \\
	\hline
	$10^{-12}$ & 5 &  0\\
	\hline
	\end{tabular}
\end{table}

\begin{table}[ht]
	\centering
	\small
	\begin{tabular}{| c | c | c | c | c|}
	\hline
	Tolleranza & Iterazioni & Newton Modificato \\
	\hline
	 $10^{-3}$  & 3 &-1.570796335324655e+00 \\
	\hline
	$10^{-6}$ & 4 & -1.570796356741072e+00 \\
	\hline
	$10^{-9}$ & 148 &  -1.570796314458764e+00\\
	\hline
	$10^{-12}$ & Il metodo non converge &  Il metodo non converge\\
	\hline
	\end{tabular}
\end{table}

Il metodo di Aitken non è corretto, per cui non è stato preso in considerazione per i paragoni. Possiamo notare che il metodo di newton modificato è più efficiente rispetto al metodo di newton  normale, che inizia a "faticare" quando aumenta la molteplicità della radice.