\documentclass[12pt]{article}

\usepackage{amsmath}

\begin{document}
\title{Elaborato Calcolo numerico 2020\\  Autori: Emanuele Brizzi, Massimo Hong}
\maketitle
\newpage
\tableofcontents
\cleardoublepage

\section{Esercizio1}

Verificare che, per h sufficientemente piccolo, si ha \\

\begin{equation*}
\frac{f(x-h)-2f(x)+f(x+h)}{h^2} = O(h^2)
\end{equation*}
Per la dimostrazione utilizziamo il polinomio di taylor di grado n centrato in $x_0:$

\begin{gather*}
P_n(x;x_0) = \sum_{k=0}^{n}\frac{(x-x_0)^k}{k!}f^{(k)}(x_0)\\
R_n(x;x_0) = O(x-x_0)^{(n+1)}\\
\end{gather*}
Per cui possiamo calcolare $f(x+h)$ e $f(x-h)$ sviluppando il polinomio di Taylor centrato in x: \\
\begin{gather*}
f(x+h) = f(x)+hf'(x)+\frac{h^2}{2}f''(x)+\frac{h^3}{6}f'''(x)+O(h^4)\\
f(x-h) = f(x)-hf'(x)+\frac{h^2}{2}f''(x)-\frac{h^3}{6}f'''(x)+O(h^4)
\end{gather*}
Sostituendo all'equazione iniziale otteniamo: \\
\begin{equation*}	
\frac{f(x-h)-2f(x)+f(x+h)}{h^2}  = \frac{(h^2)f''(x)+O(h^4)}{h^2}=  f''(x)+O(h^2)
\end{equation*}



\section{Esercizio2}
Spiegare cosa calcola il seguente script Matlab:
u = 1; while 1, if 1+u==1, break, end, u = u/2; end, u

Teoricamente questo script non dovrebbe mai terminare poichè dividendo per 2 il valore u, non si dovrebbe mai raggiungere lo 0, tuttavia il seguente script termina. Più precisamente, alla fine dell'esecuzione  risulta che $u = 1.1102\cdot10^-16$ , che è minore del valore $d = eps = 2.2204\cdot10^-16$(che rappresenta la distanza da 1.0 al valore in doppia precisione immediatamente successivo) e quando andiamo a sommare ad un numero n un valore u<eps, non verrà effettuata alcuna modifica sul valore.

\section{Esercizio3}
\begin{enumerate}
\item a = 1e20; b = 100; a-a+b
\item a = 1e20; b = 100; a+b-a
\end{enumerate}
Spiegare i risultati ottenuti.
\begin{enumerate}
\item a= 1e20; b = 100; a-a+b \\
Questo script restituisce il valore 100, in quanto $a-a = 0$ e $0+100 = 100$.
\item a= 1e20; b = 100; a+b-a
Matlab ha il valore $ eps = 2.2204\cdot10^-16$ , che corrisponde a circa 15-16 cifre di precisione ( che 1e20 supera facilmente), per cui quando andiamo ad affettuare $a+b = c$, otterremo un valore approssimato c con le prime 15-16 cifre equivalenti ad a, e di conseguenza $c-a = 0$.
\end{enumerate}

\section{Esercizio 4}
A quanto pare si deve usare la formula di newton (Modificandola un po').

\end{document}